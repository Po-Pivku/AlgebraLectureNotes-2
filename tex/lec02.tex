\makeatletter
\def\input@path{{../}}
\makeatother
\documentclass[../main.tex]{subfiles}

\begin{document}
\begin{remark}
    Очевидно композиция гомоморфизмов --- гомоморфизм.
\end{remark}

\section{Смежные классы}
Возьмем любую группу $G$ и любую подгруппу $H$ в ней. Введем следующее отношение: пусть $g_1, g_2 \in G$, тогда $g_2 \sim g_1$, если $\exists \, h \in H\colon g_2 = g_1 \cdot h$.

\begin{statement}
    $\sim$ --- отношение эквивалентности.
\end{statement}
\begin{proof}
    \begin{equation*}
        \begin{gathered}
            g = g \cdot e, \, e \in H \text{ --- рефлексивность}\\
            g_2 = g_1 \cdot h \underset{\mathclap{h^{-1} \in H}}{\implies} g_1 = g_2 \cdot h^{-1}
            \text{ --- симметричность }\\
            \begin{cases}
                g_2 = g_1 \cdot h \\
                g_3 = g_2 \cdot h'
            \end{cases}
            \underset{\mathclap{hh' \in H}}{\implies}
            g_3 = g_1 \cdot hh' \text{ --- транзитивность}
        \end{gathered}
    \end{equation*}
\end{proof}

\rmk{Обозначения}.
\begin{equation*}
    \begin{gathered}
        G/_{\sim} =: G/H\text{ --- множество классов эквивалентности по данному отношению}\\
        g \in G, M \subset G\text{. Тогда }gM = \{gm \, | \, m \in M\}\\
        g \in G, M \subset G\text{. Тогда }Mg = \{mg \, | \, m \in M\}
    \end{gathered}
\end{equation*}
Легко видеть, что $[g]$ --- класс эквивалентности элемента $g$ по отношению $\sim$, в точности равен $gH$.

\begin{definition}
    $gH$ --- \textbf{левый смежный класс}(левый класс смежности),

    $Hg$ --- \textbf{правый смежный класс}(правый класс смежности) элемента $g$ по подгруппе $H$.
\end{definition}
\begin{equation*}
    \begin{gathered}
        G/H = \{gH, \, g \in G\}\text{ --- множество всех левых смежных классов}\\
        H\backslash G = \{Hg, \, g \in G\}\text{ --- множество всех правых смежных классов}
    \end{gathered}
\end{equation*}

\begin{definition}
    Пусть $H < G$. Тогда \textbf{индексом} $H$ в $G$ называется $(G : H) = |G / H|$
\end{definition}

\begin{theorem-non}
    Количество левых смежных классов, равно количеству правых смежных классов, то есть $(G : H) = |H \backslash G|$.
\end{theorem-non}

\begin{proof}
    Построим биекцию из $G/H$ в $H \backslash G$. Пусть $A \to A^{-1} = \{a^{-1} \, | \, a \in A\}$. Это действительно отображение, потому что $A^{-1} = (gH)^{-1} = H^{-1}g^{-1} = Hg^{-1} \in H \backslash G$. В качестве обратного отображения можно также взять отображение $A \to A^{-1}$. Очевидно оно корректно определено просто по аналогии с нашей биекцией. Таким образом мы получили две взаимно обратные биекции, значит множества равномощны, что и требовалось доказать.
\end{proof}

\begin{remark}
    У конечной группы индекс очевидно всегда будет конечным. При этом у бесконечного множества все-равно может быть конечный индекс. Например $(\Z : m\Z) = |\Z/m\Z| = m$.
\end{remark}

\begin{theorem-non}
    Пусть $K < H < G$ и $(G : H) < \infty, \, (H : K) < \infty$. Тогда $(G : K) < \infty$ и $(G : K) = (G : H) \cdot (H : K)$.
\end{theorem-non}
\begin{proof}
    Пусть $(G : H) = k, \, (H : K) = l$, \:  $g_1, \dotsc, g_k$ --- представители всех классов в
$G / H$ и $h_1, \dotsc, h_l$ --- представители всех классов в $H / K$. Докажем, что $\{g_i h_j \,
| \, i = 1,\dotsc,k; \; j = 1,\dotsc, l\}$ --- представители всех классов в $G / K$. Для этого докажем следующие два утверждения:
\begin{enumerate}
    \setlength{\parskip}{0pt}
    \setlength{\itemsep}{0pt}
    \item $\forall g \in G, \, \exists \, i, j\colon g \in g_i h_j K$
    \begin{proof}
        \begin{equation*}
            \begin{gathered}
                g \in G \implies \exists \, i!\colon \; g \in g_iH \text{, т.~е. } g = g_ih, \; h \in H
            \end{gathered}
        \end{equation*}
        Аналогично
        \begin{equation*}
            h = h_jt, \; t \in K \implies g = g_i h_j t, \; t \in K
        \end{equation*}
        То есть $g \in g_ih_jK$.
    \end{proof}
    \item $g_i h_j K \neq g_{i'} h_{j'} K$ при $(i, j) \neq (i', j')$
    \begin{proof}
        Пусть $g_ih_jK = g_{i'}h_{j'}K$. Тогда $\smash{g_i\overbrace{h_j}^{\mathclap{\in H}} = g_{i'}\overbrace{h_{j'}t}^{\mathclap{\in H}}, \: t \in K}$. Но раз $g_i$ и $g_{i'}$ не могут давать один и тот же элемент если они представляют разные классы, значит они равны. То есть $i = i'$. Теперь можно на этот элемент сократить, и по той же причине получится, что $j = j'$.
    \end{proof}
\end{enumerate}
\end{proof}

\begin{examples}
    1. Пусть $G = \Z,\, H = m\Z,\, K = mn\Z$.
    \begin{equation*}
        (G : H) = m, \quad (G : K) = mn \implies (H : K) = mn/m = n
    \end{equation*}

    2. Пусть $G = S_3, H = \langle (1 2) \rangle = \{e, (1 2)\}$. Тогда:
    \begin{equation*}
        \begin{gathered}
            eH = (1 2)H = H,\\
            (1 3)H = \{(1 3), (1 2 3)\} = (1 2 3)H,\\
            (2 3)H = \{(2 3), (1 3 2)\} = (1 3 2)H\\
            H(1 3) = \{(1 3), (1 3 2)\}\\
            \cdots
        \end{gathered}
    \end{equation*}
\end{examples}

\begin{corollary*}
    Пусть $G$ --- конечная группа, $H < G$. Тогда $|G| = (G : H)|H|$.
\end{corollary*}
\begin{proof}
    Пусть $K = \{e\}$. Тогда $(G : e) = |G|, (H : e) = |H|$. Тогда по предложению
    \begin{equation*}
        (G : e) = (G : H) \cdot (H : e) \implies |G| = (G : H)|H|
    \end{equation*}
    Что и требовалось доказать.
\end{proof}

\begin{theorem}[Лагранжа]
    $|G| < \infty, \, H < G \implies |G| \dvdots |H|$
\end{theorem}
\begin{proof}
    $|G| = |H| \cdot (G : H)$
\end{proof}
\begin{corollary}
    Пусть $|G| < \infty, \, g \in G$. Тогда $|G| \dvdots \ord g $.
\end{corollary}
\begin{proof}
    $\ord g = |\langle g \rangle|$. При этом $\langle g \rangle$ --- подгруппа $G$. Значит по теореме Лагранжа $|G| \dvdots |\langle g \rangle| = \ord g$
\end{proof}
\begin{corollary}
    Пусть $|G| < \infty, \, g \in G$. Тогда $g^{|G|} = e$.
\end{corollary}
\begin{proof}
    $|G|$ делится на $\ord g$. Значит $g^{|G|} = g^{k \cdot \ord g} = e^k = e$
\end{proof}
\begin{corollary}[теорема Эйлера]
    Пусть $m \in \N, \, a \in \Z, \, (a, m) = 1$. Тогда $a^{\varphi(m)} = 1 \pmod{m}$.
\end{corollary}
\begin{proof}
    Пусть $G = (\Z /m \Z)^{*}, \, g = [a]_{m}$. Так как $(a, m) = 1$, то класс $a$ будет элементом группы $(\Z/m\Z)^{*}$. Тогда по следствию 2
    \begin{equation*}
        ([a]_m)^{|(\Z/m\Z)^{*}|} = [1]_m
    \end{equation*}
    Заметим, что $|\Z/m\Z^{*}| = \varphi(m)$ --- просто потому что число обратимо по модулю $m$ тогда и только тогда, когда их $\GCD$ равен единице. Но тогда
    \begin{equation*}
        ([a]_m)^{\varphi(m)} = [1]_m \implies a^{\varphi(m)} \equiv 1 \pmod{m}
    \end{equation*}
    Что и требовалось доказать.
\end{proof}

Вспомним как мы вводили операцию умножения на множестве $\Z/m\Z\colon [a]_m \cdot [b]_m = [a \cdot b]_m$. Попробуем обобщить это на случай левых смежных классов. Пусть:
\begin{equation*}
    g_1H \times g'H = gg'H
\end{equation*}
Для доказательства корректности заданной операции проверим, что результат операции не изменяется от выбора представителя в классе. Заменим $g$ на $gh, \, h \in H$ и $g'$ на $g'h', \, h' \in H$:
\begin{align*}
        ghg'\overbrace{h'H}^{\mathclap{H}} &\overset{\mathclap{?}}{=} gg'H\\
        hg'H &\overset{\mathclap{?}}{=} g'H\\
        (g')^{-1}hg'H &\overset{\mathclap{?}}{=} H = eH\\
        (g')^{-1}hg' &\overset{\mathclap{?}}{\in} H
\end{align*}
Мы будем определять наше умножение только для подгрупп, удовлетворяющих получившемуся утверждению. Речь о них пойдет в следующем параграфе.

\section{Нормальные подгруппы}
\begin{definition}
    Пусть $H$ --- подгруппа в $G$.\, $H$ называется \textbf{нормальной подгруппой}, если $\forall g \in G, \, \forall h \in H\colon ghg^{-1} \in H$.
\end{definition}
\rmk{Обозначение}. $H \triangleleft G \iff H$ --- нормальная подгруппа $G$.
\begin{remark}
    $G$ --- абелева $\implies$ любая подгруппа $G$ --- нормальная.
\end{remark}

\begin{theorem-non}
    Пусть $\varphi\colon G \to G'$ --- гомоморфизм, тогда $\Ker \varphi \triangleleft G$.
\end{theorem-non}
\begin{proof}
    Пусть $h \in \Ker \varphi, \, g \in G$. Тогда:
    \begin{equation*}
        \varphi(ghg^{-1}) = \varphi(g) \overbrace{\varphi(h)}^{\mathclap{e}} \varphi(g)^{-1} = e
        \implies ghg^{-1} \in \Ker \varphi
    \end{equation*}
\end{proof}

\begin{theorem-non}
    Пусть $H < G$. Тогда следующие утверждения эквивалентны:
    \begin{enumerate}
        \item $H \triangleleft G$.
        \item $\forall g \in G\colon gHg^{-1} \subset H$.
        \item $\forall g \in G\colon gHg^{-1} = H$.
        \item $\forall g \in G\colon Hg = gH$
     \end{enumerate}
\end{theorem-non}
\begin{proof}
    $1 \Leftrightarrow 2\colon \quad gHg^{-1} = \{ghg^{-1} \, | \, h \in H\}$ --- оба утверждения о том, что это множество должно лежать в $H \implies$ они эквивалентны.

    $3 \Rightarrow 2\colon$
    \begin{equation*}
        gHg^{-1} = H \implies gHg^{-1} \subset H
    \end{equation*}

    $2 \Rightarrow 3\colon \quad \forall g \in G\colon$
    \begin{equation*}
        \left.
        \begin{gathered}
            gHg^{-1} \subset H \\
            g^{-1}Hg \subset H \implies H \subset gHg^{-1}
        \end{gathered}
        \right \}
        \implies gHg^{-1} = H
    \end{equation*}

    $3 \Rightarrow 4\colon$
    \begin{equation*}
        gHg^{-1} = H \implies gH = Hg\text{ --- домножили на $g$}
    \end{equation*}

    $4 \Rightarrow 3\colon$
    \begin{equation*}
        Hg = gH \implies H = gHg^{-1}\text{ --- домножили на $g^{-1}$}
    \end{equation*}
\end{proof}

\begin{examples}
    1) $\langle (1 2) \rangle$ --- не нормальная подгруппа в $S_3$.

    2) $A_3$(четные перестановки в $S_3$) --- нормальная группа в $S_3$. Это так потому что у перестановок $g$ и $g^{-1}$ одинаковая четность, а значит четная перестановка останется четной.(ровно по этой причине множество $S_3 \backslash A_3$ тоже нормальная подгруппа)
\end{examples}
\begin{remark}
    $(G : H) = 2 \implies H \triangleleft G$
\end{remark}
\begin{proof}
    \begin{equation*}
        \begin{gathered}
            G / H = \{H, G \backslash H\}\\
            H \backslash G = \{H, G \backslash H\}
        \end{gathered}
    \end{equation*}
\end{proof}

\begin{remark}
Очевидно, пересечение набора нормальных подгрупп --- нормальная подгруппа.
\end{remark}

Пусть $H \triangleleft G$. Введем на $G / H$ структуру группы.
\begin{equation*}
    \begin{gathered}
        G/H \times G/H \to G/H \\
        (A, B) \mapsto AB
    \end{gathered}
\end{equation*}
Проверим, что $AB \in G/H$. Пусть $A = gH, \, B = g'H$. Тогда $AB = gHg'H = gg'HH = (gg')(HH) = gg'H \in G/H$.

\begin{theorem-non}
    $(G / H, \, \cdot)$ --- группа.
\end{theorem-non}
\begin{proof}
    1) Ассоциативность очевидна, потому что верна ассоциативность для множеств. $A, B, C \subset G \implies (AB)C = A(BC)$

    2) $eH$ --- нейтральный в $G / H$.
    \begin{equation*}
        \begin{gathered}
            gH \cdot eH = gHH = gH\\
            eh \cdot gH = HgH = gHH = gH
        \end{gathered}
    \end{equation*}

    3) $g^{-1}H$ --- обратный к $gH$.
    \begin{equation*}
        \begin{gathered}
            g^{-1}HgH = g^{-1}gH = eH \\
            gHg^{-1}H = eH
        \end{gathered}
    \end{equation*}
\end{proof}
\begin{definition}
    $G / H$ называется \textbf{факторгруппой} группы $G$ по нормальной подгруппе $H$.
\end{definition}

\begin{theorem-non}
    1. Факторгруппа $\Z/m\Z$ --- совпадает с ранее определенной $\Z/m\Z$.

    2. $G / G$ --- тривиальная группа(все элементы попадают в один класс).

    3. $G / \{e\} \cong G$
\end{theorem-non}

\begin{definition}
    Элементы $g$ и $g'$ называются \textbf{сравнимыми} по нормальной подгруппе $H$, если
    \begin{equation*}
      gH = g'H \iff g' = gh, \, h \in H \iff g' = \tilde{h}g, \, \tilde{h} \in H
    \end{equation*}
\end{definition}

\begin{definition}
    $ghg^{-1}$ называется \textbf{сопряженным} к $h$ при помощи $g$.
\end{definition}

\end{document}
