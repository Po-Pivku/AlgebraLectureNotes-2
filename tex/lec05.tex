\makeatletter
\def\input@path{{../}}
\makeatother
\documentclass[../main.tex]{subfiles}

\begin{document}

\section{Центр и коммутант группы}
\begin{definition}
  Пусть $G$ --- группа. Тогда \textbf{центром} группы называется
  \begin{equation*}
    \mathcal{Z}(G) = \{g \in G \, | \, \forall h \in G\colon gh = hg\}
  \end{equation*}
  Элементы центра называются \textbf{центральными} элементами группы $G$.
\end{definition}
\begin{theorem-non}
  $\mathcal{Z}$ --- это нормальная подгруппа $G$.
\end{theorem-non}
\begin{proof}
  \begin{enumerate}
    \item Замкнутость относительно умножения очевидна, $hg_1g_2 = g_1hg_2 = g_1g_2h$.
    \item Замкнутость относительно взятия обратного --- $g^{-1}h = (h^{-1}g)^{-1} = (gh^{-1})^{-1} = hg^{-1}$.
    \item Нормальность группы --- $g \in \mathcal{Z}(G) \implies h^{-1}gh = gh^{-1}h = g \in \mathcal{Z}(G)$.
  \end{enumerate}
\end{proof}

\begin{definition}
  Пусть $g, h \in G$. Тогда \textbf{коммутатором} этих двух элементов называется элемент $[g, h] = ghg^{-1}h^{-1}$.
\end{definition}
\begin{remark}
  \begin{enumerate}
    \item $[g, h] = e \iff gh = hg$. Доказательство тривиально --- $ghg^{-1}h^{-1} = e \iff gh = hg$(соответственно умножаем или делим на $hg$). Как следствие: $gh = [g, h]hg$.
    \item Обратимый к коммутатору элемент сам является коммутатором: $[g, h]^{-1} = [h, g]$.
  \end{enumerate}
\end{remark}

\begin{definition}
  \textbf{Коммутантом} группы $G$ называется подгруппа, порожденная всевозможными коммутаторами:
  \begin{equation*}
    [G, G] \coloneqq \lan [g, h] \, | \, g,h \in G \ran
  \end{equation*}
\end{definition}

\begin{theorem-non}
\label{non:6.18}
  $[G, G]$ --- нормальная подгруппа $G$.
\end{theorem-non}
\begin{proof}
  \begin{enumerate}
    \item $[G, G]$ --- подгруппа по определению.
    \item Пусть $a, g, h \in G$ и $g^a \coloneqq aga^{-1}$. Очевидно, что $(g_1g_2)^a = ag_1g_2a^{-1} = ag_1a^{-1}ag_2a^{-1} = g_1^ag_2^a$. Тогда легко видеть, что
    \begin{equation*}
      \begin{gathered}
        G \longrightarrow G\\
        g \mapsto g^a
      \end{gathered}
    \end{equation*}
    --- это автоморфизм группы $G$(изоморфизм группы на себя). Отсюда
    \begin{equation*}
      [g, h]^a = g^ah^a(g^{-1})^a(g^{-1})^a =
      g^ah^a(g^a)^{-1}(h^a)^{-1} = [g^a, h^a]
    \end{equation*}
    То есть применение сопряжения к коммутатору дает коммутатор. Тогда рассмотрим сопряжение любого элемента из $[G, G]$:
    \begin{equation*}
      ([g_1, h_1] \dotsm [g_m, h_m])^a
      =
      [g_1,h_1]^a \dotsm [g_m, h_m]^a
      =
      [g_1^a, h_1^a] \dotsm [g_m^a, h_m^a]
      \in [G, G]
    \end{equation*}
  \end{enumerate}
\end{proof}

\begin{theorem-non}
\label{non:6.19}
  Пусть $H \triangleleft G$. Тогда следующие утверждения эквивалентны:
  \begin{enumerate}
    \item $G/H$ абелева.
    \item $[G, G] \subset H$.
  \end{enumerate}
\end{theorem-non}
\begin{proof}
  \begin{align*}
    &G/H\text{ абелева } \\
    &\iff
    \forall g_1, g_2 \in G\colon (g_1H)(g_2H) = (g_2H)(g_1H) \\
    &\iff
    \forall g_1, g_2 \in G\colon (g_1H)(g_2H)(g_1H)^{-1}(g_2H)^{-1} = eH \\
    &\iff
    \forall g_1, g_2 \in G\colon (g_1g_2g_1^{-1}g_2^{-1})H = eH \\
    &\iff
    \forall g_1, g_2 \in G\colon [g_1, g_2] \in H \\
    &\iff
    [G, G] \subset H
  \end{align*}
\end{proof}

\begin{definition}
  Группа $G$ называется \textbf{разрешимой}, если существует цепочка подгрупп
  \begin{equation*}
    \{e\} = G_n < \dotsb G_2 < G_1 < G_0 = G
  \end{equation*}
  такая, что $\forall k =1..n,\; G_{k + 1} \triangleleft G_k$ и фактор $G_k/G_{k + 1}$ --- абелев.
\end{definition}
\begin{remark}
  Все абелевы группы очевидно будут разрешимыми(можно взять $G_0 = G,\, G_1 = \{e\}$).
\end{remark}

\begin{example}
  $D_n$ --- не абелева, но разрешимая группа симметрий $n$-угольника. Можно взять $\{e\} < \lan R_1 \ran < D_n$, где $\lan R_1 \ran$ --- подгруппа поворотов(подгруппа порожденная единичным поворотом). Тогда $\lan R_1 \ran / \{e\}$ очевидно абелев, потому что $\lan R_1 \ran$ --- абелева. И $D_n / \lan R_1 \ran \cong \Z/2\Z$, а значит $\lan R_1 \ran$ --- нормальная подгруппа и фактор $D_n / \lan R_1 \ran$ абелев.
\end{example}

\rmk{Обозначение.}
Пусть
\begin{equation*}
  G^{(k)} \coloneqq
  \begin{cases}
    G, &k = 0 \\
    [G^{(k - 1)}, G^{(k - 1)}], &k \geq 1
  \end{cases}
\end{equation*}

\begin{theorem}
  Пусть $G$ --- группа. Тогда следующие утверждения эквивалентны:
  \begin{enumerate}
    \item $G$ разрешима.
    \item $\exists \, n \in \N\colon G^{(n)} = \{e\}$.
  \end{enumerate}
\end{theorem}
\begin{proof}
  \begin{enumerate}
    \item \textbf{``2 $\Rightarrow$ 1''.} Возьмем цепочку $\{e\} = G^{(n)} < \dotsb < G^{(1)} < G^{(0)} = G$. Тогда из того, что $G^{(k + 1)}$ --- коммутант $G^{(k)}$ следует, что $G^{(k + 1)} \triangleleft G^{(k)}$ по предложению \ref{non:6.18} и $G^{(k)}/G^{(k + 1)}$ --- абелев по предложению \ref{non:6.19}.
    \item \textbf{``1 $\Rightarrow$ 2''.} $G$ --- разрешимая, значит существует цепочка подгрупп
    \begin{equation*}
      \begin{gathered}
        \{e\} = G_n < \dotsb G_1 < G_0 = G \\
        G_{k + 1} \triangleleft G_k, \; G_k/G_{k + 1}\text{ абелев}
      \end{gathered}
    \end{equation*}
    Докажем по индукции: $G^{(k)} \subset G_k$. База $k = 0\colon G^{(0)} = G = G_0$. Тогда
    \begin{equation*}
      G^{(k + 1)} = [G^{(k)}, G^{(k)}] \overbrace{\subset}^{\mathclap{\text{
        инд. предположение
      }}} [G_k, G_k]
      \overbrace{\subset}^{\mathclap{\text{
        \ref{non:6.19}
      }}}
      G_{k + 1}
    \end{equation*}
    Отсюда в частности $G^{(n)} \subset G_n = \{e\} \implies G^{(n)} = \{e\}$.
  \end{enumerate}
\end{proof}

\begin{remark}
  Если $G$ --- конечная группа, то выполнено хотя бы одно из двух условий:
  \begin{enumerate}
    \item $\exists\, n\colon G^{(n)} = \{e\}$.
    \item $\exists\, n\colon G^{(n)} = G^{(n - 1)} \neq \{e\}$.
  \end{enumerate}
\end{remark}

\begin{definition}
  Группа называется \textbf{простой}, если она не имеет нормальных подгрупп, отличных от всей группы и единичной подгруппы.
\end{definition}

\section{Действие группы на множестве}
\begin{definition}
\label{def:6.20}
  Пусть $G$ --- группа, $M$ --- множество. Говорят, что \textbf{задано действие $G$ на множестве $M$}, если задано отображение $G \times M \to M, \; (g, m) \mapsto gm$ с двумя свойствами:
  \begin{enumerate}
    \item $\forall g_1, g_2 \in G, \, \forall m \in M\colon (g_1g_2)m = g_1(g_2m)$
    \item $\forall m \in M\colon em = m$
  \end{enumerate}
\end{definition}
\begin{examples}
\label{ex:6.20}
  \begin{enumerate}
    \item Пусть $G = \R$, $M = \C$ и $gm = m \cdot (\cos g + i \sin g)$. Геометрически это означает поворот точки $m$ на координатной плоскости на угол $g$.
    \item Пусть $G = \GL_n(K)$, $M = K^n$. Тогда $g = A$ --- матрица, $m = b$ --- столбец, $gm = Ab$.
    \item Пусть $G$ --- любая группа, $*$ --- операция в $G$, $M = G$. Тогда $gm \coloneqq g * m$ --- \textbf{действие $G$ на себе левыми сдвигами}.
    \item Пусть $G$ --- любая группа, $*$ --- операция в $G$, $M = G$. Тогда $gm \coloneqq gmg^{-1}$ --- \textbf{действие $G$ на себе сопряжением}.
  \end{enumerate}
\end{examples}

\begin{definition}
\label{def:6.21}
  Говорят, что \textbf{задано действие группы $G$ на множестве $M$}, если задан гомоморфизм $\varphi\colon G \to S(M)$.
\end{definition}
\begin{theorem-non}
  Равносильность двух определений.
\end{theorem-non}
\begin{proof}
  \begin{enumerate}
    \item \textbf{``1 $\Rightarrow$ 2''.} Пусть задано отображение $(g, m) \mapsto gm$. Тогда определим $\varphi\colon G \to S(M)$ как $\varphi(g)~=~(m \mapsto gm)$. Получилась биекция, так как существует обратное отображение $m \mapsto g^{-1}m$. Докажем, что $\varphi$ --- гомоморфизм.
    \begin{equation*}
      \varphi(g_1g_2)(m) = (g_1g_2)(m) = g_1(g_2m) = \varphi(g_1)(\varphi(g_2)(m))
    \end{equation*}
    Таким образом $\varphi(g_1g_2) = \varphi(g_1) \circ \varphi(g_2)$.
    \item \textbf{``2 $\Rightarrow$ 1''.}
    Рассмотрим отображение
    \begin{equation*}
      \begin{gathered}
        G \times M \to M \\
        (g, m) \mapsto \varphi(g)(m)
      \end{gathered}
    \end{equation*}
    Тогда
    \begin{equation*}
      \begin{gathered}
        (g_1g_2)(m) = (\varphi(g_1) \circ \varphi(g_2))(m) = g_1(g_2m) \\
        em = \varphi(e)(m) = \id_m(m) = m
      \end{gathered}
    \end{equation*}
  \end{enumerate}
\end{proof}

\begin{definition}
  Пусть задано $G$ на $M$ и $m \in M$. Тогда \textbf{орбитой} элемента $m$ называют множество
  \begin{equation*}
    Gm = \{gm \, | \, g \in G\}
  \end{equation*}
\end{definition}

\begin{examples}
  Найдем орбиты элементов из примеров к определению \ref{def:6.20}.
  \begin{enumerate}
    \item $Gm = \{z \in \C \, | \, |z| = |m|\}$.
    \item $Gm =
    \begin{cases}
      \{0\}, m = 0 \\
      K^n \, \backslash\, \{0\}, m \neq 0
    \end{cases}$
    \item $Gm = M = G$.
    \item Орбиты --- классы сопряженности.
  \end{enumerate}
\end{examples}

\begin{remark}
  Орбиты двух элементов $M$ либо не пересекаются, либо совпадают.
\end{remark}
\begin{proof}
  Пусть $n \in Gm_1 \cap Gm_2$. Тогда $n = g_1m_1,\; g_1 \in G$ и $n = g_2m_2,\; g_2 \in G$. Отсюда $m_2 = g_2^{-1}g_1m_1 \in Gm_1 \implies \forall g \in G\colon gm_2 = gg_2^{-1}g_1m_2 \in Gm_1 \implies Gm_2 \subset Gm_1$. Аналогично получается, что $Gm_1 \subset Gm_2$.
\end{proof}
\begin{exercise}
  Передоказать замечание, используя отношение эквивалентности.
\end{exercise}

\begin{definition}
  Если $\exists m\colon Gm = M$, то действие $G$ на $M$ называют \textbf{транзитивным}.
\end{definition}

\begin{definition}
  Пусть $G$ действует на $M$ и $m \in M$. \textbf{Стабилизатором(стационарной подгруппой)} элемента $m$ называется подгруппа
  \begin{equation*}
    \St_m = \{g \in G \, | \, gm = m\}
  \end{equation*}
\end{definition}

\begin{theorem-non}
  \begin{enumerate}
    \item $St_m < G$.
    \item Существует биекция $G/St_m \overset{\alpha}{\longrightarrow} Gm$ такая, что $g \cdot St_m \mapsto gm$.
  \end{enumerate}
\end{theorem-non}
\begin{proof}
  \begin{enumerate}
    \item Пусть $g, g_1, g_2 \in St_m$. Тогда
    \begin{equation*}
      \begin{gathered}
        (g_1g_2)m = g_1(g_2m) = g_1m = m \\
        g^{-1}m = g^{-1}(gm) = (g^{-1}g) m = em = m
      \end{gathered}
    \end{equation*}
    Доказали замкнутость относительно умножения и взятия обратного.
    \item Пусть $g_1St_m  = g_2St_m \implies g_2 = g_1h, \, h \in St_m \implies g_2m = (g_1h)m = g_1(hm) = g_1m$. Осталось проверить биективность. Пусть
    \begin{equation*}
      \begin{gathered}
        \alpha(g \cdot \St_m) = \alpha(g' \cdot \St_m) \implies \\
        gm = g'm \implies \\
        m = g^{-1}g'm \implies \\
        g^{-1}g' \in \St_m \implies \\
        g' \in g \cdot \St_m \implies \\
        g'\St_m = g\St_m
      \end{gathered}
    \end{equation*}
    Инъективность доказана, а сюръективность тривиальна.
  \end{enumerate}
\end{proof}
\begin{corollary*}
  Пусть $G$ --- конечная группа, $G$ действует на $M$ и $m \in M$. Тогда $|Gm|$ делит $|G|$.
\end{corollary*}
\begin{proof}
  По предложению $|Gm| = |G/\St_m|$, но $|G/\St_m| = (G : \St_m)$, что делит $G$.
\end{proof}

\begin{examples}
  Снова обратимся к примерам для определения \ref{def:6.20} и найдем стабилизаторы в заданных действиях.
  \begin{enumerate}
    \item $\St_m =
    \begin{cases}
      \R, &m = 0 \\
      \lan 2\pi \ran, &m \neq 0
    \end{cases}$
  \item Упражнение для читателя.
  \item $\St_g = \{e\}$.
  \item $\St_g = \{h \in G \, | \, hg = gh\} = \mathcal{Z}_g$ --- централизатор $g$.
  \end{enumerate}
\end{examples}

\begin{theorem-non}
  Пусть $G$ --- конечная $p$-группа(т.е. $|G| = p^n,\, p$ --- простое). Тогда $\mathcal{Z}(G)\, \neq\, \{e\}$.
\end{theorem-non}
\begin{proof}
  Рассмотрим действие $G$ на себе сопряжением. Тогда $G$ --- это объединение орбит, при этом порядок любой орбиты является делителем порядка группы. Тогда возможные порядки орбит это $1, p, p^2, \dotsc, p^n$.

  Пусть $m_i$ --- число орбит длины $p^i$. Тогда
  \begin{equation*}
    p^n = 1 \cdot m_0 + \underbrace{p \cdot m_1 + p^2 \cdot m_2 + \dotsb + p^n \cdot m_n}_{\mathclap{\text{
      делится на $p$
    }}}
  \end{equation*}
  Отсюда следует, что $m_0$ делится на $p$. Но элементы, которые под действием сопряжения переходят в себя --- это центральные элементы группы. Таким образом $m_0 = |\mathcal{Z}(G)| \implies p \, | \, |\mathcal{Z}(G)| \implies \mathcal{Z}(G) \neq \{e\}$.
\end{proof}

\begin{corollary*}
  Любая конечная $p$-группа разрешима.
\end{corollary*}
\begin{proof}
  Пусть $|G| = p^n$, индукция по $n$. Рассмотрим $G' = G / \mathcal{Z}(G)$. Из того, что $\mathcal{Z}(G) \neq \{e\}$ следует, что $|G'| < |G|$. С другой стороны очевидно,  что $|G'| = (G : \mathcal{Z}(G)) = \frac{|G|}{|\mathcal{Z}(G)|}$, а значит $G'$ --- $p$-группа.

  Если $G' = \{e\} \implies G = \mathcal{Z}(G) \implies G$ --- абелева, а значит разрешима. Иначе $G' \neq \{e\}$, а значит по индукционному предположению существует цепочка подгрупп $\{e\} = G_n' < \dotsb < G_0' = G'$, \, $G_{k + 1}' \triangleleft G_k'$ и $G_k' / G_{k + 1}'$ абелев. Рассмотрим гомоморфизм проекций на факторгруппу $\pi_{\mathcal{Z}(G)}$. Пусть $G_k \coloneqq \pi_{\mathcal{Z}(G)}^{-1}(G_k')$. Тогда рассмотрим цепочку
  \begin{equation*}
    \{e\} = G_{n + 1} < \mathcal{Z}(G) = \Ker \pi_{\mathcal{Z}(G)} = G_n < \dotsb < G_2 < G_1 < G_0 = G
  \end{equation*}
  Осталось доказать необходимые свойства нашей цепочки.
  \begin{enumerate}
    \item Заметим, что $\pi_{\mathcal{Z}(G)}\colon G_k \longrightarrow G_k'$. А значит по теореме о соответствии $G_{k + 1}' \triangleleft G_k' \implies G_{k + 1} \triangleleft G_k$.
    \item Заметим, что $G_k' \cong G_k / \mathcal{Z}(G) \implies G_k' / G_{k + 1}' \cong (G_k / \mathcal{Z}(G)) / (G_{k + 1} / \mathcal{Z}(G)) \cong G_k/G_{k + 1}$ --- абелева.
  \end{enumerate}
  Таким образом $G$ --- разрешимая.
\end{proof}

\end{document}
